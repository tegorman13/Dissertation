% Options for packages loaded elsewhere
\PassOptionsToPackage{unicode}{hyperref}
\PassOptionsToPackage{hyphens}{url}
\PassOptionsToPackage{dvipsnames,svgnames,x11names}{xcolor}
%
\documentclass[
  letterpaper,
  DIV=11,
  numbers=noendperiod]{scrartcl}

\usepackage{amsmath,amssymb}
\usepackage{iftex}
\ifPDFTeX
  \usepackage[T1]{fontenc}
  \usepackage[utf8]{inputenc}
  \usepackage{textcomp} % provide euro and other symbols
\else % if luatex or xetex
  \usepackage{unicode-math}
  \defaultfontfeatures{Scale=MatchLowercase}
  \defaultfontfeatures[\rmfamily]{Ligatures=TeX,Scale=1}
\fi
\usepackage{lmodern}
\ifPDFTeX\else  
    % xetex/luatex font selection
  \setmathfont[]{Libertinus Math}
\fi
% Use upquote if available, for straight quotes in verbatim environments
\IfFileExists{upquote.sty}{\usepackage{upquote}}{}
\IfFileExists{microtype.sty}{% use microtype if available
  \usepackage[]{microtype}
  \UseMicrotypeSet[protrusion]{basicmath} % disable protrusion for tt fonts
}{}
\makeatletter
\@ifundefined{KOMAClassName}{% if non-KOMA class
  \IfFileExists{parskip.sty}{%
    \usepackage{parskip}
  }{% else
    \setlength{\parindent}{0pt}
    \setlength{\parskip}{6pt plus 2pt minus 1pt}}
}{% if KOMA class
  \KOMAoptions{parskip=half}}
\makeatother
\usepackage{xcolor}
\usepackage[top=1in,bottom=1in,left=1in,right=1in,heightrounded]{geometry}
\setlength{\emergencystretch}{3em} % prevent overfull lines
\setcounter{secnumdepth}{-\maxdimen} % remove section numbering
% Make \paragraph and \subparagraph free-standing
\ifx\paragraph\undefined\else
  \let\oldparagraph\paragraph
  \renewcommand{\paragraph}[1]{\oldparagraph{#1}\mbox{}}
\fi
\ifx\subparagraph\undefined\else
  \let\oldsubparagraph\subparagraph
  \renewcommand{\subparagraph}[1]{\oldsubparagraph{#1}\mbox{}}
\fi


\providecommand{\tightlist}{%
  \setlength{\itemsep}{0pt}\setlength{\parskip}{0pt}}\usepackage{longtable,booktabs,array}
\usepackage{calc} % for calculating minipage widths
% Correct order of tables after \paragraph or \subparagraph
\usepackage{etoolbox}
\makeatletter
\patchcmd\longtable{\par}{\if@noskipsec\mbox{}\fi\par}{}{}
\makeatother
% Allow footnotes in longtable head/foot
\IfFileExists{footnotehyper.sty}{\usepackage{footnotehyper}}{\usepackage{footnote}}
\makesavenoteenv{longtable}
\usepackage{graphicx}
\makeatletter
\def\maxwidth{\ifdim\Gin@nat@width>\linewidth\linewidth\else\Gin@nat@width\fi}
\def\maxheight{\ifdim\Gin@nat@height>\textheight\textheight\else\Gin@nat@height\fi}
\makeatother
% Scale images if necessary, so that they will not overflow the page
% margins by default, and it is still possible to overwrite the defaults
% using explicit options in \includegraphics[width, height, ...]{}
\setkeys{Gin}{width=\maxwidth,height=\maxheight,keepaspectratio}
% Set default figure placement to htbp
\makeatletter
\def\fps@figure{htbp}
\makeatother
% definitions for citeproc citations
\NewDocumentCommand\citeproctext{}{}
\NewDocumentCommand\citeproc{mm}{%
  \begingroup\def\citeproctext{#2}\cite{#1}\endgroup}
\makeatletter
 % allow citations to break across lines
 \let\@cite@ofmt\@firstofone
 % avoid brackets around text for \cite:
 \def\@biblabel#1{}
 \def\@cite#1#2{{#1\if@tempswa , #2\fi}}
\makeatother
\newlength{\cslhangindent}
\setlength{\cslhangindent}{1.5em}
\newlength{\csllabelwidth}
\setlength{\csllabelwidth}{3em}
\newenvironment{CSLReferences}[2] % #1 hanging-indent, #2 entry-spacing
 {\begin{list}{}{%
  \setlength{\itemindent}{0pt}
  \setlength{\leftmargin}{0pt}
  \setlength{\parsep}{0pt}
  % turn on hanging indent if param 1 is 1
  \ifodd #1
   \setlength{\leftmargin}{\cslhangindent}
   \setlength{\itemindent}{-1\cslhangindent}
  \fi
  % set entry spacing
  \setlength{\itemsep}{#2\baselineskip}}}
 {\end{list}}
\usepackage{calc}
\newcommand{\CSLBlock}[1]{\hfill\break#1\hfill\break}
\newcommand{\CSLLeftMargin}[1]{\parbox[t]{\csllabelwidth}{\strut#1\strut}}
\newcommand{\CSLRightInline}[1]{\parbox[t]{\linewidth - \csllabelwidth}{\strut#1\strut}}
\newcommand{\CSLIndent}[1]{\hspace{\cslhangindent}#1}

\KOMAoption{captions}{tableheading}
\makeatletter
\@ifpackageloaded{caption}{}{\usepackage{caption}}
\AtBeginDocument{%
\ifdefined\contentsname
  \renewcommand*\contentsname{Table of contents}
\else
  \newcommand\contentsname{Table of contents}
\fi
\ifdefined\listfigurename
  \renewcommand*\listfigurename{List of Figures}
\else
  \newcommand\listfigurename{List of Figures}
\fi
\ifdefined\listtablename
  \renewcommand*\listtablename{List of Tables}
\else
  \newcommand\listtablename{List of Tables}
\fi
\ifdefined\figurename
  \renewcommand*\figurename{Figure}
\else
  \newcommand\figurename{Figure}
\fi
\ifdefined\tablename
  \renewcommand*\tablename{Table}
\else
  \newcommand\tablename{Table}
\fi
}
\@ifpackageloaded{float}{}{\usepackage{float}}
\floatstyle{ruled}
\@ifundefined{c@chapter}{\newfloat{codelisting}{h}{lop}}{\newfloat{codelisting}{h}{lop}[chapter]}
\floatname{codelisting}{Listing}
\newcommand*\listoflistings{\listof{codelisting}{List of Listings}}
\makeatother
\makeatletter
\makeatother
\makeatletter
\@ifpackageloaded{caption}{}{\usepackage{caption}}
\@ifpackageloaded{subcaption}{}{\usepackage{subcaption}}
\makeatother
\ifLuaTeX
  \usepackage{selnolig}  % disable illegal ligatures
\fi
\IfFileExists{bookmark.sty}{\usepackage{bookmark}}{\usepackage{hyperref}}
\IfFileExists{xurl.sty}{\usepackage{xurl}}{} % add URL line breaks if available
\urlstyle{same} % disable monospaced font for URLs
\hypersetup{
  colorlinks=true,
  linkcolor={DarkSlateBlue},
  filecolor={Maroon},
  citecolor={DarkSlateBlue},
  urlcolor={DarkSlateBlue},
  pdfcreator={LaTeX via pandoc}}

\author{}
\date{}

\begin{document}
\begin{centering}
\LARGE
{The Role of Variability in Learning Transfer: A Similarity-Based Computational Approach}

 
\vspace*{1.5cm}

\LARGE
{Thomas E. Gorman}

\vspace{16.5cm}

\end{centering}

Submitted to the faculty of the University Graduate School in partial
fulfillment of the requirements for the degree Doctor of Philosophy in
the Department of Psychology and Brain Sciences and the Cognitive
Science Program, Indiana University Indiana University

\vspace{6cm}

\pagenumbering{gobble}

\newpage

Accepted by the Graduate Faculty, Indiana University, in partial
fulfillment of the requirements for the degree of Doctor of Philosophy.
\vspace{4cm}

\hfill\break
\_\_\_\_\_\_\_\_\_\_\_\_\_\_\_\_\_\_\_\_\_\_\_\_\_\_\_\_\_ Robert L.
Goldstone, PhD \vspace{2.5cm}\\
\strut \\
\_\_\_\_\_\_\_\_\_\_\_\_\_\_\_\_\_\_\_\_\_\_\_\_\_\_\_\_\_ Robert
Nosofsky, PhD \vspace{2.5cm}\\
\_\_\_\_\_\_\_\_\_\_\_\_\_\_\_\_\_\_\_\_\_\_\_\_\_\_\_\_\_ Peter Todd,
PhD \vspace{2.5cm}\\
\_\_\_\_\_\_\_\_\_\_\_\_\_\_\_\_\_\_\_\_\_\_\_\_\_\_\_\_\_ Mike Jones,
PhD

\newpage

\begin{centering}

\vspace*{6.5cm}

@2023 \\
\vspace{1cm} 

Thomas E. Gorman
\vspace{.2cm}

\vspace{5cm}

\end{centering}

\newpage
\begin{center}
\textbf{Acknowledgements}
\end{center}
\newpage

\newpage{}

\section{Abstract}\label{abstract}

This dissertation seeks to explore the cognitive underpinnings that
govern the generalization of learning, focusing specifically on the role
of variability during training in shaping subsequent transfer
performance. A comprehensive review of the existing literature is
presented, emphasizing the methodological complications associated with
disentangling the confounding effects of similarity. Through a series of
experiments involving several novel visuomotor tasks, this work
investigates whether and how variability in training conditions affects
performance in novel tasks. To theoretically account for the empirical
outcomes, I employ both instance-based and connectionist computational
models, both of which incorporate similarity-based mechanisms. These
models serve to account for the extent to which variability influences
the learners' generalization gradient, and also explain how training
variation can produce both beneficial and deleterious outcomes.

\newpage{}

\tableofcontents
\newpage
\listoffigures
\newpage
\listoftables
\newpage

\newpage{}

\subsection{Main body}\label{main-body}

Following the procedure used by Mcdaniel et al.
(\citeproc{ref-mcdanielPredictingTransferPerformance2009}{2009}), we
will assess the ability of both ALM and EXAM to account for the
empirical data when fitting the models to 1) only the training data, and
2) both training and testing data. Models will be fit directly to the
trial by trial data of each individual participants, both by minimizing
the root-mean squared deviation (RMSE), and by maximizing log
likelihood. Because ALM has been shown to do poorly at accounting for
human patterns extrapolation
(\citeproc{ref-deloshExtrapolationSineQua1997}{DeLosh et al., 1997}), we
will also fit the extended EXAM version of the model, which operates
identically to ALM during training, but includes a linear extrapolation
mechanism for generating novel responses during testing.

quarto pandoc --citeproc --pdf-engine xelatex -t pdf\\
--bibliography=../Assets/Bib/Dissertation.bib\\
--standalone\\
-f markdown igas\_e1.pdf.md\\
-o refer-test.pdf

quarto render igas\_e1.qmd --citeproc --pdf-engine xelatex -t pdf\\
--bibliography=../Assets/Bib/Dissertation.bib\\
--standalone\\
-o refer-test.pdf

\section{Appendix}\label{appendix}

\subsubsection{Appendix - Project 2 - Experiment
1}\label{appendix---project-2---experiment-1}

\subsection{E1 Appendix}\label{e1-appendix}

\subsubsection{Posterior Predictive
Distributions}\label{posterior-predictive-distributions}

\subsubsection{Empirical vs.~Predicted}\label{empirical-vs.-predicted}

\subsubsection{Different Aggregations}\label{different-aggregations}

\section*{References}\label{references}
\addcontentsline{toc}{section}{References}

\phantomsection\label{refs}
\begin{CSLReferences}{1}{0}
\bibitem[\citeproctext]{ref-deloshExtrapolationSineQua1997}
DeLosh, E. L., McDaniel, M. A., \& Busemeyer, J. R. (1997).
Extrapolation: {The Sine Qua Non} for {Abstraction} in {Function
Learning}. \emph{Journal of Experimental Psychology: Learning, Memory,
and Cognition}, \emph{23}(4), 19.
\url{https://doi.org/10.1037/0278-7393.23.4.968}

\bibitem[\citeproctext]{ref-mcdanielPredictingTransferPerformance2009}
Mcdaniel, M. A., Dimperio, E., Griego, J. A., \& Busemeyer, J. R.
(2009). Predicting transfer performance: {A} comparison of competing
function learning models. \emph{Journal of Experimental Psychology.
Learning, Memory, and Cognition}, \emph{35}, 173--195.
\url{https://doi.org/10.1037/a0013982}

\end{CSLReferences}



\end{document}
